%%%%%%%%%%%%%%%%%%%%%%%%%%%%%%%%%%%%%%%%%
% University of Calgary Solar Car Team 
% LaTeX Template
% Version 1.0 (23/10/19)
%
% This template has been downloaded from:
% http://www.LaTeXTemplates.com
%
% Original author:
% WikiBooks (http://en.wikibooks.org/wiki/LaTeX/Title_Creation)
%
% License:
% CC BY-NC-SA 3.0 (http://creativecommons.org/licenses/by-nc-sa/3.0/)
% 
% Instructions for using this template:
% This title page is capable of being compiled as is. This is not useful for 
% including it in another document. To do this, you have two options: 
%
% 1) Copy/paste everything between \begin{document} and \end{document} 
% starting at \begin{titlepage} and paste this into another LaTeX file where you 
% want your title page.
% OR
% 2) Remove everything outside the \begin{titlepage} and \end{titlepage} and 
% move this file to the same directory as the LaTeX file you wish to add it to. 
% Then add \input{./title_page_1.tex} to your LaTeX file where you want your
% title page.
%
%%%%%%%%%%%%%%%%%%%%%%%%%%%%%%%%%%%%%%%%%
%\title{Title page with logo}
%----------------------------------------------------------------------------------------
%	PACKAGES AND OTHER DOCUMENT CONFIGURATIONS
%----------------------------------------------------------------------------------------

\documentclass[12pt]{report}
\usepackage[english]{babel}
\usepackage[margin=1in]{geometry}
\usepackage[utf8x]{inputenc}
\usepackage{amsmath}
\usepackage{float}
\usepackage[nottoc,numbib]{tocbibind}
\usepackage{graphicx}
\usepackage[colorinlistoftodos]{todonotes}
\usepackage{hyperref}
  \usepackage[tocentry]{vhistory}
\usepackage{pdfpages}
\usepackage{appendix}

%Table of contents links
\hypersetup{
    colorlinks=true, %set true if you want colored links
    linktoc=all,     %set to all if you want both sections and subsections linked
    linkcolor=black,  %choose some color if you want links to stand out
}

\begin{document}

\begin{titlepage}

\newcommand{\HRule}{\rule{\linewidth}{0.5mm}} % Defines a new command for the horizontal lines, change thickness here

\center % Center everything on the page
 
%----------------------------------------------------------------------------------------
%	HEADING SECTIONS
%----------------------------------------------------------------------------------------

\textsc{\LARGE University of Calgary Solar Car Team}\\[1.5cm] % Name of your university/college
\textsc{\Large Electrical Documentation}\\[0.5cm] % Major heading team name

%----------------------------------------------------------------------------------------
%	TITLE SECTION
%----------------------------------------------------------------------------------------

\HRule \\[0.4cm]
  { \huge \bfseries Auxiliary Battery Management System (Aux-BMS)}\\[0.4cm] % Title of your document
\HRule \\[1.5cm]
 
%----------------------------------------------------------------------------------------
%	AUTHOR SECTION
%----------------------------------------------------------------------------------------

\Large \emph{Original Author:}\\
Chathula Adkary \\[3cm] % Your name

%----------------------------------------------------------------------------------------
%	DATE SECTION
%----------------------------------------------------------------------------------------

{\large \today}\\[2cm] % Date, change the \today to a set date if you want to be precise

%----------------------------------------------------------------------------------------
%	LOGO SECTION
%----------------------------------------------------------------------------------------

\includegraphics[width=\textwidth]{../../Images/Logos/logo-wide.png}\\[1cm] % Include a department/university logo - this will require the graphicx package
 
%----------------------------------------------------------------------------------------

\vfill % Fill the rest of the page with whitespace

\end{titlepage}

%----------------------------------------------------------------------------------------
%       TABLE OF CONTENTS
%----------------------------------------------------------------------------------------

\tableofcontents
%\listoffigures
%\listoftables
%\newpage

%----------------------------------------------------------------------------------------
%       INTRODUCTION
%----------------------------------------------------------------------------------------

\chapter{Introduction}

\section{About}

\chapter{Initial Testing of Rev. v1.0}
\section{Overcurrent Protection Circuitry - Logic Testing}
\subsection{Comparator Operation}
  \begin{itemize}
        \item \textbf{Equipment Used:} Rigol DP832A PSU, Rigol DS1054Z Oscilloscope, BlackMagic Probe
        \item \textbf{Test Procedure:} Current monitor jumper was disconnected from comparator, and a voltage was manually applied to simulate the analog voltage representing current draw. The potentiometer was set to 1.53V. When a voltage of less than or equal to 1.49V was applied to the analog curent signal (no current sense amp), no change was observed. When a voltage of 1.49V or greater was applied to the analog curent signal (no current sense amp), the red LED turned on, showing an overcurrent event. The voltage was reduced below 1.4V, and the red LED stayed on. This is expected. The overcurrent reset button was pressed. This caused the red LED to turn off. Applied voltage remained at 1.4V. Then, the voltage was raised above 1.49V. The red LED turned on, as before. When the voltage was still above 1.49V, the overcurrent reset button was pressed. This had no effect on the red LED. It remained on. This is expected behaviour.
        \item \textbf{Conclusion:} Overcurrent protection appears to be functioning properly. It must be test with actual current loads.
  \end{itemize}
\subsection{Contactor ON/OFF}
  \begin{itemize}
        \item \textbf{Equipment Used:} Rigol DP832A PSU, Rigol DS1054Z Oscilloscope, BlackMagic Probe
        \item \textbf{Test Procedure:} Connect the 5V and GND lines of the UART connector to channel 1 of a benchtop PSU to power the logic and MCU. Current monitor jumper was disconnected from comparator, and a voltage was manually applied to simulate the analog voltage representing current draw. The MCU was programmed to output '1' on PE2 (CONTACTOR\_ENABLE2). If the logic is working correctly, the green LED to indicate that CONTACTOR B is enabled should turn on, if the red overcurrent LED is off.
        \item \textbf{Conclusion:} The overcurrent protection circuitry is working as expected. When the red LED (overcurrent) is on, no contactors may turn on (all contactor LEDs are off). When the red LED is off (no overcurrent condition), the contactors may turn on, controlled by the MCU. This was tested with no load attached to the contactor control circuitry. The next step is to test the current amplifier with a real load, and see if the overcurrent circuitry performs as expected. 
  \end{itemize}
\section{Current Amplifier Validation}
Your introduction goes here! Some examples of commonly used commands and features are listed below, to help you get started.
f you have a question, please use the support box in the bottom right of the screen to get in touch. 

\bibliographystyle{IEEEtran}
\bibliography{ref}

\begin{versionhistory}
  \vhEntry{1.0}{23.10.2019}{CA}{Created}
\end{versionhistory}

\begin{appendices}
  \chapter{Schematics}
\end{appendices}

\end{document}
