%%%%%%%%%%%%%%%%%%%%%%%%%%%%%%%%%%%%%%%%%
% University of Calgary Solar Car Team 
% LaTeX Template
% Version 1.0 (23/10/19)
% % This template has been downloaded from:
% http://www.LaTeXTemplates.com
%
% Original author:
% WikiBooks (http://en.wikibooks.org/wiki/LaTeX/Title_Creation)
%
% License:
% CC BY-NC-SA 3.0 (http://creativecommons.org/licenses/by-nc-sa/3.0/)
% 
% Instructions for using this template:
% This title page is capable of being compiled as is. This is not useful for 
% including it in another document. To do this, you have two options: 
%
% 1) Copy/paste everything between \begin{document} and \end{document} 
% starting at \begin{titlepage} and paste this into another LaTeX file where you 
% want your title page.
% OR
% 2) Remove everything outside the \begin{titlepage} and \end{titlepage} and 
% move this file to the same directory as the LaTeX file you wish to add it to. 
% Then add \input{./title_page_1.tex} to your LaTeX file where you want your
% title page.
%
%%%%%%%%%%%%%%%%%%%%%%%%%%%%%%%%%%%%%%%%%
%\title{Title page with logo}
%----------------------------------------------------------------------------------------
%	PACKAGES AND OTHER DOCUMENT CONFIGURATIONS
%----------------------------------------------------------------------------------------

\documentclass[12pt]{article}
\usepackage[english]{babel}
\usepackage[margin=1in]{geometry}
\usepackage[utf8x]{inputenc}
\usepackage{amsmath}
\usepackage{float}
\usepackage[nottoc,numbib]{tocbibind}
\usepackage{graphicx}
\usepackage{booktabs}
\usepackage[colorinlistoftodos]{todonotes}
\usepackage{hyperref}
\usepackage[tocentry]{vhistory}
\usepackage{appendix}
\usepackage{pgfplotstable}  %To generate a table from .csv

%Table of contents links
\hypersetup{
    colorlinks=true, %set true for colored links
    linktoc=all,     %set to have both sections and subsections linked
    linkcolor=black,  %choose some color if you want links to stand out
    urlcolor=brown,
}

\pgfplotsset{compat=1.15}

\begin{document}

  \begin{titlepage}
  
    \newcommand{\HRule}{\rule{\linewidth}{0.5mm}} % Defines a new command for the horizontal lines, change thickness here
    
    \center % Center everything on the page
     
    %----------------------------------------------------------------------------------------
    %	HEADING SECTIONS
    %----------------------------------------------------------------------------------------
    
    \textsc{\LARGE University of Calgary Solar Car Team}\\[1.5cm] % Name of your university/college
    \textsc{\Large Electrical Documentation}\\[0.5cm] % Major heading team name
    
    %----------------------------------------------------------------------------------------
    %	TITLE SECTION
    %----------------------------------------------------------------------------------------
    
    \HRule \\[0.4cm]
      { \huge \bfseries Title}\\[0.4cm] % Title of your document
    \HRule \\[1.5cm]
     
    %----------------------------------------------------------------------------------------
    %	AUTHOR SECTION
    %----------------------------------------------------------------------------------------
    
    \Large \emph{Original Author:}\\
    Your Name \\[3cm] % Your name
    
    %----------------------------------------------------------------------------------------
    %	DATE SECTION
    %----------------------------------------------------------------------------------------
    
    {\large \today}\\[2cm] % Date, change the \today to a set date if you want to be precise
    
    %----------------------------------------------------------------------------------------
    %	LOGO SECTION
    %----------------------------------------------------------------------------------------
    
    \includegraphics[width=\textwidth]{../../Images/Logos/logo-wide.png}\\[1cm] % Include a department/university logo - this will require the graphicx package
     
    %----------------------------------------------------------------------------------------
    
    \vfill % Fill the rest of the page with whitespace
  
  \end{titlepage}
  
  %----------------------------------------------------------------------------------------
  %       TABLE OF CONTENTS
  %----------------------------------------------------------------------------------------
  
  \tableofcontents
  \listoffigures
  \listoftables
  %\newpage
  
  %----------------------------------------------------------------------------------------
  %       INTRODUCTION
  %----------------------------------------------------------------------------------------
  
  \section{Introduction}
  Provide a brief introduction to the Circuit Board and its purpose within the car. 
  
  \section{Inputs/Outputs from the Board}
  Provide a high level description of the connections to the connectors of the board.
  
  \section{Pinout and Pin Description of Major Components}

  \subsection{Component A}
  \begin{table}[H]
    \begin{center}
      \caption{Table with pins and pin descriptions for Component A}
      \label{tab:ComponentA}
      \pgfplotstabletypeset[
        col sep=comma, % the seperator in our .csv file
        display columns/0/.style={
                  column name=$\textbf{Pin Number}$, 
                  column type={|c}
                },
        display columns/1/.style={
                  column name=$\textbf{Pin Name}$,
                  string type,
                  column type={|c|}
                },
        display columns/2/.style={
                  column name=$\textbf{Description}$,
                  string type,
                  column type={c|}
                },
        every head row/.style={before row={\hline}, after row=\hline},
        every odd row/.style={after row=\hline},
        every even row/.style={after row=\hline},
        every last row/.style={after row=\hline},
      ]{ComponentATable.csv} % filename/path to file
    \end{center}
  \end{table}

  \subsection{Component B}
  \begin{table}[H]
    \begin{center}
      \caption{Table with pins and pin descriptions for Component B}
      \label{tab:ComponentA}
      \pgfplotstabletypeset[
        col sep=comma, % the seperator in our .csv file
        display columns/0/.style={
                  column name=$\textbf{Pin Number}$, 
                  column type={|c}
                },
        display columns/1/.style={
                  column name=$\textbf{Pin Name}$,
                  string type,
                  column type={|c|}
                },
        display columns/2/.style={
                  column name=$\textbf{Description}$,
                  string type,
                  column type={c|}
                },
        every head row/.style={before row={\hline}, after row=\hline},
        every odd row/.style={after row=\hline},
        every even row/.style={after row=\hline},
        every last row/.style={after row=\hline},
      ]{ComponentBTable.csv} % filename/path to file
    \end{center}
  \end{table}

  \section{Functional Description}
  Provide a thorough description of what the purpose of the Board is.

  \section{Testing}
  Provide a thorough description of the testing procedures that you used to ensure proper functioning of the board before putting it in the car.
  
  \bibliographystyle{IEEEtran}
  \bibliography{ref}
  
  \begin{versionhistory}
    \vhEntry{1.0}{23.10.2019}{(Initials)}{Description}
  \end{versionhistory}

\end{document}
